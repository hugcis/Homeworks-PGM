\documentclass[12pt, oneside]{amsart}   	% use "amsart" instead of "article" for AMSLaTeX format
\usepackage[top=1.3cm,left=1.9cm, right=1.9cm, bottom=1.3cm]{geometry}                		% See geometry.pdf to learn the layout options. There are lots.
\geometry{letterpaper}                   		% ... or a4paper or a5paper or ... 
%\geometry{landscape}                		% Activate for rotated page geometry
%\usepackage[parfill]{parskip}    		% Activate to begin paragraphs with an empty line rather than an indent
\usepackage{graphicx}				% Use pdf, png, jpg, or eps§ with pdflatex; use eps in DVI mode
\usepackage{bm}							% TeX will automatically convert eps --> pdf in pdflatex		
\usepackage{amssymb}
\usepackage{amsmath}
\usepackage{subcaption}
\usepackage{float}
\pagenumbering{gobble}
%SetFonts

\newcommand\indep{\protect\mathpalette{\protect\indeP}{\perp}}
 \def\indeP#1#2{\mathrel{\rlap{$#1#2$}\mkern2mu{#1#2}}}


%SetFonts


\title{}
\author{}
\date{}							% Activate to display a given date or no date

\begin{document}
\section{Conditional independence and factorizations}
\textbf{1.}  The implied factorization of a distribution $p\in \mathcal{L}(G)$ is $p(x, y, z, t) = p(t|z)p(z|x, y)p(x)p(y)$. For two real valued independent r.v $X$ and $Y$ that follow two distribution $\mathcal{U}(0, 2)$, and $Z = X+Y$, $T = \mathbf{1}_{(Z \geq 3)}$, the statement $X\indep Y | T$ is not true.
Indeed, knowing that $X+Y \geq 3$ makes the two random variables not independent ($p(x| y=2, z=1) = p(x \geq 1) \neq p(x|z=1)$).
\vspace{0.5cm}
 
 \textbf{2.a.} If $Z$ is a binary variable, let $p = p(z=1) = 1 - p(z=0)$. We have 
 \begin{align*}
 	p(x, y) = p(x)p(y) &= p(x, y|z=1)p + p(x, y|z=0)(1-p) & (X\indep Y ) \\
	& = p(x|z=1)p(y|z=1)p + p(x|z=0)p(y|z=0)(1-p) & (X\indep Y | Z)\\
	&  = p(x)p(y)\left(\frac{1}{p} p(z=1|x)p(z=1|y) + \frac{1}{1-p}p(z=0|x)p(z=0|y)\right)
 \end{align*}
 If we write $p_x = p(z=1|x) = 1-p(z=0|x)$ and $p_y = p(z=1|y)$, the equality becomes $p^2 - (p_x + p_y)p + p_xp_y
 = 0$ which yields $ \left(p(z=1) = \right)\ p = p_x\ ( =  p(z=1|x) )$ or $p = p_y$. Therefore either $X \indep Z$ or $Y \indep Z$.
\vspace{0.5cm}
 
 \textbf{2.b.} Let $\mathcal{A}$ be a finite space, $X_0$ and $Y_0$ two independent random variables on $\mathcal{A}$ , 
 we define $X =\begin{bmatrix} X_0 \\ 0 \end{bmatrix}$, $Y = \begin{bmatrix} 0 \\ Y_0 \end{bmatrix}$ and $Z = \begin{bmatrix} X_0 \\ Y_0 \end{bmatrix}$. 
 
 $X$, $Y$ and $Z$ are three random variables with $X \indep Y$ and $ X \indep Y | Z$
 
 
 \section{Distributions factorizing in a graph}
 \textbf{1.} Let $p\in\mathcal{L}(G)$, $p(x) = \prod_{s\in V} p(x_s|x_{\pi_s})$. Since $\{i\rightarrow j\}$ is a covered edge,  $\pi_j = \pi_i \cup \{i\}$. Therefore,
 \begin{align*}
 p(x) = p(x_i|x_{\pi_i})\cdot p(x_j|x_{\pi_i}, x_i) \cdot\prod_{s\in V-\{ i, j \}} p(x_s|x_{\pi_s})
 \end{align*}
 And 
 \begin{align*}
 p(x_i|x_{\pi_i})\cdot p(x_j|x_{\pi_i}, x_i) &= \frac{p(x_i, x_{\pi_i})}{p(x_{\pi_i})} \frac{p(x_i, x_{\pi_i}, x_j)}{p(x_{\pi_i}, x_i)} =p(x_i,  x_j |x_{\pi_i})\cdot \frac{p(x_j, x_{\pi_i})}{p(x_j, x_{\pi_i})}\\
 & = p(x_i | x_{\pi_i}, x_j)\cdot p(x_j|x_{\pi_i})
 \end{align*}
 In $G'$, $\{i\rightarrow j\}$ has been replaced by $\{j\rightarrow i\}$, thus $\pi_i' = \pi_i \cup \{j\}$ and $\pi_j' = \pi_i $, the equation above shows that $p$ can be factorized in this new graph, hence $p\in \mathcal{L}(G')$. From the same equality above, if $p\in \mathcal{L}(G')$, also $p\in \mathcal{L}(G)$. Therefore $\boxed{ \mathcal{L}(G) =  \mathcal{L}(G')}$.
 \vspace{0.5cm}
 
 \textbf{2.} Since $G$ is a directed tree, it doesn't contain any v-structure and all nodes have either 0 or 1 parent (only the root of the tree doesn't have a parent). Moreover, all cliques or $G'$ contain at most 2 elements because any clique of 3 or more elements would contain a 3-clique. That 3-clique in the directed graph must contain either a v-structure or a cycle, which is not possible in a directed tree. 
 
 Hence if $p\in \mathcal{L}(G)$, $p(x) = \prod_i p(x_i | x_{\pi_i}) = \prod_i \Psi_i(x_i, x_{\pi_i})$, where $\Psi_i$ is a function of an element and its parent (a 2-clique of the graph), or of the root of the tree only (a 1-clique). We have $\boxed{\mathcal{L}(G) \subset \mathcal{L}(G')}$.
 
 
 \end{document}